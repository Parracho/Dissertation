\chapter{Introduction}







\section{}



\section{Structure outline}

The first chapter of this work, this introduction, aims to establish the scene for the following content and to give an overview of the subject.

In the second chapter we first study the $3+1$ covariant formalism. This formalism creates the framework through which the spacetime manifold can be decomposed into a slicing of spacelike hypersurfaces, each perpendicular to a chosen timelike vector. Secondly, we check the conformal counterparts of the dynamical quantities of the formalism.

The third chapter is to present a general covariant and gauge invariant averaging procedure applied to the past light-cone of an observer and to introduce the correspoding Buchert-Ehlers commutation rules. 

The fourth chapter is dedicated to the study and discussion of modified theories of gravity, mainly Brans-Dicke theories, where we introduce a new scalar field, the Brans-dicke fluid.

The fifth chapter is devoted initially to present the general Buchert equations, where we underline the various types of backreaction, followed by applying the previously obtained formalisms and averaging procedure to the Brans-Dicke theory, in order to determine the equivalent Friedmann equations.


\section{Notation and conventions}
\begin{itemize}
    \item Spacetime indices are represented by Greek letters and adquire values from 0 to 3, i.e. $\mu,\nu,\lambda,\ldots - 0,1,2,3$.
    \item Spatial indices are indicated by Latin letters starting from $i$ onwards and run from 1 to 3, i.e. $i,j,k,\ldots - 1,2,3$.
    \item Angular indices are shown by Latin letter starting from $a$ until $g$ and run from 2 to 3, i.e. $a,b,c,\ldots - 2,3$.
    \item Einstein summation convenction is used: repeated index implies summation over all the possible values of the index.
    \item Tensors which are represented in negrit are equal to, for example: for vectors to their linear form by $\mathbf{v}=v^{(index)}\partial_{(index)}$ as well as their dual form by $\mathbf{v}=v_{(index)}dx^{(index)}$.
    \item The signature of the metric is (-+++).
    \item Natural units are employed, to the extent that $c=1$. (Como vamos trabalhar com brans-dicke isto precisa de mais certeza em relacao a G)
    \item Equality by definition is indicated by $:=$.
    \item Equality by algebraic identity is represented by $\equiv$.
    \item Symmetrization is represented by round brackets, for example: $T_{(\mu\nu)}:=\frac{1}{2}(T_{\mu\nu}+T_{\nu\mu})$.
    \item Antisymmetrization is represented by square brackets, for example: $T_{[\mu\nu]}:=\frac{1}{2}(T_{\mu\nu}-T_{\nu\mu})$.
    \item Traceless tensors are represented by angle brackets, for example: $T_{\langle\mu\nu\rangle}:=T_{(\mu\nu)}-\frac{1}{3}g_{\mu\nu}(g^{\lambda\rho}T_{\lambda\rho})$.
    \item The partial derivative w.r.t $x^\mu$ is indicated by $\partial_\mu$ or with a comma, for example: $u_{\mu,\nu}=\partial_\nu u_\mu$.
    \item The covariant derivative w.r.t $x^\mu$ is indicated by $\nabla_\mu$ or with a semicolon, for example: $u_{\mu;\nu}=\nabla_\nu u_\mu$.
\end{itemize}