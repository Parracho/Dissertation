\chapter{Formalism}

\section{Eulerian velocity}
\label{app:eulerian}


To determine this vector, let us first consider a fluid element at a point $p\in \Sigma_t$.
Let $\tau$ be the Eulerian observer's \footnote{The Eulerian observer is the observer where it's velocity is the unit timelike vector $\mathbf{n}$ in Fig.\ref{fig:foliation}} proper time at $p$.
After a coordinate time $t+dt$, the fluid element as moved to the point $q\in \Sigma_{t+dt}$. Locally the event $q'$, which for the observer happens at a time $\tau + d\tau$, is given by the orthogonal projection of $q$ onto the observer's worldline\footnote{The space of simultaneous events for the Eulerian observer is the space orthogonal to his 4-velocity $\mathbf{n}$.}.
Defining the infinitesimal distance between $q$ and $q'$ as $d\mathbf{l}$. Then $d\tau_0$ be the proper time increment of the fluid, this time is related with the proper time of the observer by the Lorentz factor, $\Gamma$,
\begin{equation}
    d\tau := \Gamma d\tau_0
    \label{eqn:time_dilation}
\end{equation}

We also get the triangle identity,
\begin{equation}
    d\tau_0 \mathbf{u} = d\tau \mathbf{n} + d\mathbf{l}
    \label{eqn:triangle_identity}
\end{equation}
by taking the scalar product with $\mathbf{n}$,
\begin{equation}
    d\tau_0 \mathbf{u} \cdot \mathbf{n} = d\tau \underbrace{\mathbf{n} \cdot \mathbf{n}}_{-1} + \underbrace{d\mathbf{l} \cdot \mathbf{n}}_{0}
\end{equation}
hence with the relation in Eq.(\ref{eqn:time_dilation}),
\begin{equation}
    \Gamma=-\mathbf{u} \cdot \mathbf{n}
\end{equation}
with respect to the coordinates $(t,x^i)$,
\begin{equation}
    \Gamma = Nu^0
\end{equation}

The fluid velocity relative to the Eulerian observer is defined as,
\begin{equation}
    \mathbf{V}=\frac{d\mathbf{l}}{d\tau}.
    \label{eqn:definition_eulerian_fluid_velocity}
\end{equation}

By dividing the identity \ref{eqn:triangle_identity} by $d\tau$ and using Eq.\ref{eqn:time_dilation},
\begin{equation}
    \mathbf{u}=\Gamma(\mathbf{n}+\mathbf{V}).
    \label{eqn:general_fluid_expression}
\end{equation}



\section{Gauss Relation}
\label{app:gauss}

From the Ricci identity in \cref{eqn:spatial_ricci_identity}



\section{Codazzi Relation}
\label{app:codazzi}


By using the Ricci identity in the spacetime manifold,
\begin{equation}
    \left[\nabla_\mu,\nabla_\nu\right]w^\lambda=R^\lambda{}_{\rho\mu\nu}w^\rho
\end{equation} 
where $\mathbf{w}$ is a spacetime vector field.




\section{Ricci Relation}
\label{app:ricci}



